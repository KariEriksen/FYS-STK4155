\documentclass[a4paper, twocolumn]{article}

% Included packages ---------------------------------------------------------- %
\usepackage{lipsum}                          % Generate random, blind, filler-text.
\usepackage[utf8]{inputenc}                  % utf-8 encoding, æ, ø , å, etc.
\usepackage{a4wide}                          % Adjust margins to better fit A4 format.
\usepackage{array}                           % Matrices.
\usepackage{dsfont}                          % Math symbols.
\usepackage{amsmath}                         % Math symbols, and enhanced matrices.
\usepackage{amsfonts}                        % Math fonts.
\usepackage{amssymb}                         % Additional symbols.
\usepackage{mathrsfs}                        % Most additional symbols.
\usepackage[pdftex]{graphicx}                % Improved inclusion of .pdf-graphics files.
\usepackage{sidecap}                         % Floats with captions to the right/left.
\usepackage{enumerate}                       % Change counters (arabic, roman, etc.).
\usepackage{floatrow}                        % Multi-figure floats.
\usepackage{subfig}                          % Multi-figure floats.
\usepackage{bm}                              % Bolded text in math mode.
\usepackage[framemethod=default]{mdframed}   % Make boxes.
\usepackage{listings}                        % For including source code.
\usepackage{mathtools}                       % Underbrackets, overbrackets.
\usepackage[dvipsnames]{xcolor}              % Colors.
\usepackage{capt-of}                         % Caption things which are not floats.
\usepackage{algorithm2e}                     % Algorithm non-float which we can caption by \captionof{algocf}{<caption>}.
\usepackage{fontawesome}                     % Github icon, etc. \faGithub
\usepackage{sidecap}                         % Floats with captions on the side.
\usepackage{tabularx}                        % Tables and stuff.
\usepackage{tabulary}                        % Tables and stuff.
\usepackage[sf,sl,outermarks]{titlesec}      % Change fonts in section{}, subsection{}, etc.
\usepackage[subfigure]{tocloft}              % Change spacing between numbers and titles in TOC.
\usepackage{pgfplots}                        % Making Tikz plots.
\usepackage{booktabs}                        % \toprule, \midrule, etc. for tables.
\usepackage{siunitx}                         % Allows S table column, aligning on decimal point.
\usepackage{chngcntr}                        % Change counter behaviour, supress increment of sub counters.
\usepackage{tikz}                            % Draw complicated pictures in a super hard way.
\usepackage{cuted}                           % Place long equations in @twocolumn over entire page.
\usepackage[version=4]{mhchem}               % Chemical reaction equations using \ce{...}.
\usepackage[%                                % Adds functionality to captions.
  tableposition = top,
  labelsep      = period,
  justification = raggedright,
  format        = hang,
  ]{caption}                                 
\usepackage[%                                % Interactive references and links, colored.
  colorlinks  = true,
  linkcolor   = black,
  urlcolor    = blue,
  citecolor   = black,
  linktocpage = true,
  ]{hyperref}            
\usepackage[%                                % References, in super-script form.
  autocite    = superscript,
  backend     = biber,
  sortcites   = true,
  style       = numeric-comp,
  sorting     = none,
  url         = false,
  ]{biblatex}
\usepackage[autostyle, english = american]{csquotes} % Assure quotation marks are inserted correctly aligned left/right.
\MakeOuterQuote{"}

% Package settings ----------------------------------------------------------- %
\renewcommand{\thesection}{\Roman{section}}         % I, II, III, IV, etc. section numbering
\renewcommand{\thesubsection}{\Alph{subsection}}    % A, B, C, etc. subsection numbering
\renewcommand{\thesubsubsection}{}                  % Remove subsubsection numbering.
\floatsetup[table]{capposition=top}                 % Place table captions above the table.
\captionsetup[subfigure]{labelformat=empty}         % Remove the (a), (b), etc. tags from subfigures.
\advance\cftsecnumwidth 1.0em\relax                 % Set the spacing between section headings and titles in TOC with tocloft.
\advance\cftsubsecindent 1.0em\relax                % Set the spacing between subsection headings and titles in TOC with tocloft.
\advance\cftsubsecnumwidth 1.0em\relax              % Set the spacing between subsubsection headings and titles in TOC with tocloft.
\newcommand{\listingsfont}{\ttfamily}
\newcommand{\inlinepy}[1]{\lstinline[language={python}]{#1}}
\newcommand{\inlinecc}[1]{\lstinline[language={c++}]{#1}}
\counterwithout*{subsection}{section}               % Dont reset the subsection counter on new \section{} calls.
\renewcommand{\figurename}{FIG.}                    % Captions of figures read FIG.
\renewcommand{\tablename}{TABLE}                    % Captions of tables read TABLE 
\renewcommand{\thetable}{\Roman{table}}             % Number tables with roman numerals.
\usetikzlibrary{matrix}                             % Some tikz picture library things.
\usetikzlibrary{arrows.meta, calc, chains, positioning}
\definecolor{listingsbackgroundcolor}{rgb}{0.975,0.975,0.975}
\colorlet{shadecolor}{listingsbackgroundcolor}



% Section headings settings -------------------------------------------------- %
\titleformat{\section}[hang]  % {command}[shape]
  {\normalfont\bfseries}      % {format}
  {\thesection.}              % {label}
  {2ex}                       % {sep}
  {\centering\MakeUppercase}  % {before-code}[after-code]

\titleformat{\subsection}[hang] % {command}[shape]
  {\normalfont\bfseries}        % {format}
  {\thesubsection.}             % {label}
  {1ex}                         % {sep}
  {\centering}                  % {before-code}[after-code]

\titleformat{\subsubsection}[hang]  % {command}[shape]
  {\normalfont\bfseries}            % {format}
  {}                                % {label}
  {1ex}                             % {sep}
  {\centering}                      % {before-code}[after-code]


% References ----------------------------------------------------------------- %
\newcommand{\Fig}[1]{Fig.\ \ref{fig:#1}}
\newcommand{\fig}[1]{Fig.\ \ref{fig:#1}}
\newcommand{\eq} [1]{Eq.\ (\ref{eq:#1})}
\newcommand{\Eq} [1]{Eq.\ (\ref{eq:#1})}
\newcommand{\tab}[1]{Table \ref{tab:#1}}
\newcommand{\Tab}[1]{Table \ref{tab:#1}}

% Matrices ------------------------------------------------------------------- %
\newcommand{\mat} [2]{\begin{matrix}[#1] #2 \end{matrix}}    % Nothing enclosing it.
\newcommand{\pmat}[2]{\begin{pmatrix}[#1] #2 \end{pmatrix}}  % Enclosing parentheses.
\newcommand{\bmat}[2]{\begin{bmatrix}[#1] #2 \end{bmatrix}}  % Enclosing square brackets.
\newcommand{\vmat}[2]{\begin{vmatrix}[#1] #2 \end{vmatrix}}  % Enclosing vertical bars.
\newcommand{\Vmat}[2]{\begin{Vmatrix}[#1] #2 \end{Vmatrix}}  % Enclosing double bars.

% Manually set alignment of rows / columns in matrices (mat, pmat, etc.) ----- %
\makeatletter
\renewcommand*\env@matrix[1][*\c@MaxMatrixCols c]{%
  \hskip -\arraycolsep
  \let\@ifnextchar\new@ifnextchar
  \array{#1}}
\makeatother

% figures in multicols environment ------------------------------------------- %
\newenvironment{Figure}
  {\par\medskip\noindent\minipage{\linewidth}}
  {\endminipage\par\medskip}

% Set bibliography file and path for images.
\addbibresource{../ref/project3-references.bib}
\bibliography{../ref/project3-references.bib}
\graphicspath{{../figures/}}

% Black frame with gray background ------------------------------------------ %
\definecolor{gray}{gray}{0.9}
\newmdenv[linecolor=white,backgroundcolor=gray]{grayframe}
\newmdenv[linecolor=white,backgroundcolor=shadecolor]{shadeframe}

% Title
\title{{\sc Solving partial differential equations with neural networks \\ {\large FYS-STK4155: Project 3}}}
\author{Morten Ledum, Håkon Kristiansen \& Kari Eriksen \\ \faGithub \ {\small \href{https://github.com/mortele/FYS-STK4155/tree/master/project3}{github.com/mortele/FYS-STK4155}}}
% ---------------------------------------------------------------------------- %
% ---------------------------------------------------------------------------- %
\begin{document}

\onecolumn
\maketitle

\begin{abstract}
\lipsum[1]
\end{abstract}

\twocolumn
\onecolumn

\tableofcontents 
\twocolumn


\section{Introduction}
\lipsum[1]

\section{Theory}
\lipsum[2]

\subsection{The heat equation}
The heat equation is a partial differential equation (in space $x$ and time $t$) which describes the evolution of temperature differences in a region of space over a time interval. It is based on \textit{Fourier's law}; the rate of heat flow through a surface is proportional to the temperature gradient across the surface, i.e.\
\begin{align}
\mathbf{q}=-k\nabla T = -k \frac{\partial T}{\partial x},
\end{align}
where $\mathbf{q}$ denotes the heat flux density, $k$ is the thermal conductivity of the surface material, and $T$ represents the temperature. Changes in temperature are proportional to changes in internal energy, with the proportionality constant being the specific heat capacity $c_p$. With the arbitrary energy zero point placed at absolute zero, this can be written as
\begin{align}
Q = c_p \rho T,
\end{align}
with $Q$ being the internal energy and $\rho$ denoting the mass density. This is essentially just a restatement of (a shifted) \textit{first law of thermodynamics}, in the absence of applied work. The total heat energy contained in a region $[a,b]$ is given by the integral
\begin{align}
\int_a^b\mathrm{d}x\,c_p\rho T(x,t).
\end{align}
Integrating over a small region of space and considering the change in internal energy over a short time interval (assuming $c_p$ and $\rho$ are both time-independent and spatially homogenous) gives 
\begin{align}
\Delta Q=& c_p \rho\int_{x}^{x+\Delta x}\mathrm{d}\chi\,\Big[T(\chi,t+\Delta t) - T(\chi,t)\Big] \nonumber \\
&= c_p\rho \int_x^{x+\Delta x}\mathrm{d}\chi\int_{t}^{t+\Delta t}\mathrm{d}\tau\,\frac{\partial T}{\partial \tau}. \label{eq:energy1}
\end{align}
Over a short time period $\Delta t$, the change in internal energy of a short segment of length $\Delta x$  must be entirely due to the heat flux in/out of the boundaries,
\begin{align}
\Delta Q &= k\int_t^{t+\Delta t}\mathrm{d}\tau\,\left[\frac{\partial T(x+\Delta x,\tau)}{\partial x} - \frac{\partial T(x,\tau)}{\partial x} \right] \nonumber \\
%
&= k\int_t^{t+\Delta t}\mathrm{d}\tau\int_x^{x+\Delta x}\mathrm{d}\chi\,\frac{\partial^2 T}{\partial \chi^2}. \label{eq:energy2}
\end{align}
By conservation of energy, the difference between \eq{energy1} and \eq{energy2} must obviously vanish. Since we are integrating over the same spatial and temporal regions in both equations, this means that the integrand must vanish identically:
\begin{align}
\frac{k}{c_p\rho} \frac{\partial^2 T}{\partial x^2} = \frac{\partial T}{\partial t}. \label{eq:heat}
\end{align}
This is known as the \textit{heat equation} and is a special case of the more general diffusion equation. 

\subsection{Closed form solution}
The 1D heat equation may be solved by applying a separation of variables ansatz, i.e.\ we assume the solution $u(x,t)$ takes the form
\begin{align}
u(x,t) = X(x)T(t), 
\end{align}
with $X(x)$ carrying all the $x$-dependence, and $T(t)$ carrying the corresponding time dependence. Introducing the compact notation 
\begin{align}
u_{x} \equiv \partial_x u =  \frac{\partial u}{\partial x}, \ \ \text{ and } \ \ u_t \equiv \partial_t u =  \frac{\partial u}{\partial t}, 
\end{align}
we find by insertion of the ansatz into \eq{heat}:
\begin{align}
\alpha^2\, u_{xx} &= u_t \nonumber \\
%
\alpha^2\, \partial_x \Big[\partial_x  X(x)T(t) \Big] &= \partial_t X(x)T(t) \nonumber \\
%
\alpha^2\, \partial_x \Big[ X_x T + XT_x \Big] &= X_tT+XT_t \nonumber \\
%
\alpha^2\, \Big[X_{xx}T + 2X_xT_x + XT_{xx}\Big]&= XT_t \nonumber \\
\frac{1}{X(x)} \frac{\partial^2 X(x)}{\partial x^2} &= \frac{1}{\alpha^2\,T(t)}\frac{\partial T(t)}{\partial t},
\end{align}
where we defined $\alpha^2\equiv k/c_p\rho$ and used the fact that $X_t=T_x=0$. As the left hand side is independent of $t$ and the right hand side is independent of $x$, the equality can only be achieved if both sides are constant. This reduces the original partial differential equation into a set of two ordinary differential equations 
\begin{align}
\frac{1}{X(x)}\frac{\partial^2X(x)}{\partial x^2}&=k  \label{eq:X(x)}\\
%
\frac{1}{\alpha^2\,T(t)}\frac{\partial T(t)}{\partial t}&=k, \label{eq:T(t)}
\end{align} 
with $k$ an undetermined constant. 

Depending on the value of $k$, the spatial part has solutions $X(x)=Ax+B$ (if $k=0$), $X(x)=A\mathrm{e}^{\mu x}+B\mathrm{e}^{-\mu x}$ (if $k=\mu^2>0$), or $X(x)=A\mathrm{e}^{\mathrm{i}\mu x}+B\mathrm{e}^{-\mathrm{i}\mu x}$ (if $k=-\mu^2<0$). The temporal equation has solutions $T(t)=C$ (if $k=0$), or $T(t)=C\mathrm{e}^{\alpha^2\mu^2 x}$ (otherwise).

\subsubsection{Applying the boundary conditons}
In order to make progress, we need to apply the specific boundary and initial conditions. In our case, the boundaries at $x=0$ and $x=L=1$ vanish, and the initial spatial solution takes the form $u(x,t=0)=\sin \pi x$. If the $k$ constant of \eq{X(x)} vanishes, then  both constants $A$ and $B$ vanish due to the boundary conditions. The same is true if $k=\mu^2>0$. It follows that the only non-trivial solutions arise when $k=-\mu^2<0$, in which case we find (left boundary)
\begin{align}
X(0) &= A\cos \mu x + B\sin\mu x = 0 \Rightarrow A=0 \nonumber 
\end{align}
and (right boundary) 
\begin{align}
X(1) &= B\sin\mu x = 0 \Rightarrow \mu=\pi n. \nonumber 
\end{align}
This gives rise to an infinite set of equations\textemdash one for each $n$\textemdash which determine the Fourier coefficients of the initial condition $u(x,t=0)$:
\begin{align}
u(x,t=0) &= \sum_{n=1}^\infty B_n \sin (n\pi x),
\end{align}
with 
\begin{align}
B_n &= 2\int_0^1\mathrm{d}x\, u(x,t=0)\,\sin(n\pi x).
\end{align}

The temporal equation, \eq{T(t)}, can now be solved only applying the boundary conditions. With the value of $\mu$ fixed at $\mu=n\pi$, we obtain 
\begin{align}
T(t) = \mathrm{e}^{-n^2\pi^2\alpha^2 t}.
\end{align}

\subsubsection{Applying the initial condition}
Combining the $X(x)$ and $T(t)$ solutions we obtain a series representation of the solution in terms of the Fourier coefficients of the initial condition $u(x,t=0)$, in which the higher frequency modes of the initial solution decays more rapidly than the corresponding lower frequency modes,
\begin{align}
u(x,t) = \sum_{n=1}^\infty B_n \sin(n\pi x)\,\mathrm{e}^{-n^2\pi^2\alpha^2 t}. \label{eq:general}
\end{align}
It is easy to see that the steady-state solution is achieved when $u(x,t)=0$, since both boundaries (left and right) act as sinks for the initial heat energy contained in the system, and no heat is ever \textit{added}. In our case, the initial condition makes the general solution \eq{general} take on a very simple form. Applying $u(x,t=0)=\sin \pi x$, it is trivial to evaluate 
\begin{align}
B_n &= 2\int_0^1\mathrm{d}x\, \sin(\pi x)\, \sin(n\pi x) = \delta_{1n},
\end{align}
due to the orthogonality of $\sin(n\pi x)$ and $\sin(m\pi x)$. This means $B_1=1$ and $B_2,B_3,\dots = 0$,
and the full solution to \eq{heat} is given by 
\begin{align}
u(x,t)=\sin(\pi x)\,\mathrm{e}^{-\pi^2\alpha^2 t}. \label{eq:full}
\end{align}


\section{Finite difference method}
The most straightforward way to solve \eq{heat} numerically is through \textit{finite difference methods}, i.e.\ discretizing time and space on a grid and Taylor expanding the solution to obtain algebraic equation sets. A multitude of different strategies and schemes exists, but we will consider the \textit{explicit forward Euler} scheme. 

The spatial region $[0,L]$ is discretized by splitting it into $N$ segments, and considering only the functional values on the points $x_i=i\Delta x$ with $i\in[0,N-1]$. We denote a function $f(x,t)$ evaluated at $x_i$ (and at time $t$) by $f_i^t=f(x_i,t)$. Considering $N$ spatial points gives a step size $\Delta x$ between each point of 
\begin{align}
\Delta x = \frac{L}{N-1}.
\end{align}
In addition, we introduce a discretization in the temporal dimension with step size $\Delta t$. 

Let us consider the Taylor expansion of a function $f(x,t)$ considered at fixed $t$, $f(x;t)$, around a spatial point $x$. We use the shorthand $h\equiv (x-a)$, and consider the expansion at a point $a\not=x$. The expansions of $f(x+h;t)$ and $f(x-h;t)$ are given by, 
\begin{align}
f(x+h) &\approx f(x) + hf'(x) + h^2f''(x), \label{eq:f(x+h)}\\
f(x-h) &\approx f(x) - hf'(x) + h^2f''(x), \label{eq:f(x-h)}
\end{align}
respectively. As $t$ is considered a fixed parameter for the moment, we supressed the second functional argument for notational brevity. The shorthand $f'(x)$ is here used to denote differentiation w.r.t.\ $x$. Note that both equations hold with equality if an overall error term proportional to $h^3$ is added on the right hand side, i.e.\ $\mathcal{O}(h^3)$. Adding Eqs. (\ref{eq:f(x+h)}) and (\ref{eq:f(x-h)}) and dividing through by $h^2$ yields
\begin{strip}
\begin{align}
f(x+h)+f(x-h) &= 2f(x) + h^2f''(x) + \mathcal{O}(h^4) \nonumber \\
%
f''(x) &= \frac{f(x+h)-2f(x)+f(x-h)}{h^2} + \mathcal{O}(h^2), \label{eq:centraldiff}
\end{align}
\end{strip}
which is the three-point central difference approximation to the second derivative. Note that the resulting error is proportional to $h^2$ because the third order contributions from Eqs. (\ref{eq:f(x+h)}) and (\ref{eq:f(x-h)})\textemdash $h^3f'''(x+h)$ and $-h^3f'''(x+h)$\textemdash cancel exactly. A corresponding temporal first order derivative approximation may be found by simply considering \eq{f(x+h)}, and considering $h$ to be a small temporal step, $x$ to be a fixed parameter, and varying $t$. This gives
\begin{align}
f'(t)&= \frac{f(t+h)-f(t)}{h}+\mathcal{O}(h), \label{eq:forwarddiff}
\end{align}
where the error term is proportional to $h^2$ when disregarding the last term on the right hand side of \eq{f(x+h)}, which gives $\mathcal{O}(h)$ after dividing through by $h$.

Let us now consider \eq{centraldiff} on the previously defined grid, and take $h=\Delta x$. In the \eq{forwarddiff} case, we define $h=\Delta t$, and equate $\alpha^2\,f''(x)$ with $f'(t)$ as in the heat equation \eq{heat}. The result can be solved for $f(x,t+\Delta t)$, i.e.\ the \textit{next} time step given that the previous step is known:
\begin{align}
\frac{f_i^{j+1} - f_i^j}{\Delta t} &= \alpha^2\frac{f_{i+1}^j - 2f_i^j + f_{i-1}^j}{\Delta x^2} \nonumber \\
%
f_i^{j+1} &= f_i^j + \beta \Big[f_{i+1}^j - 2f_i^j + f_{i-1}^j\Big], \label{eq:explicit}
\end{align}
where $f_i^j$ denotes the discretized $f(x_i,t_j)$ and 
\begin{align}
\beta\equiv\alpha^2\left(\frac{\Delta t}{\Delta x^2}\right).
\end{align}
Equation (\ref{eq:explicit}) is known as the \textit{explicit Euler} scheme, and can be solved directly for $f(x,t+\Delta t)$ since the right hand side is all known at time step $t$.


\subsection{Solving differential equations with neural networks}
\lipsum[5]


\section{Results and discussion}
\lipsum[6]

\section{Conclusion}
\lipsum[7]


\onecolumn{
\printbibliography
}

\end{document}



